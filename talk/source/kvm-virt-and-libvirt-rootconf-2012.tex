\RHpresentationHead{
\documentclass[pdfte x,unicode,minimal,xcolor=table]{beamer}
}

\RHarticleHead{
% This does not work, because of colors, \insertauthor, etc.
\documentclass[a4paper,12pt,pdftex,unicode]{article}
\usepackage[envcountsect]{beamerarticle}
}

\usepackage[utf8x]{inputenc}

\mode<presentation> {
    \usetheme{RedHat}
    \setbeamertemplate{navigation symbols}{}
    \setbeamercovered{transparent=5}
}

\mode<article> {
    \usepackage{fullpage}
}
\mode<handout> {
    \usepackage{pgfpages}
    \pgfpagesuselayout{4 on 1}[a4paper,landscape,border shrink=5mm]
}

    
\usepackage{beamerredhat}
\usepackage{etex}
\usepackage[english]{babel}
\usepackage{setspace,amsfonts,calc,upquote,hyperref,floatflt,graphicx}
\usepackage[table]{xcolor}
\usepackage{colortbl}
\usepackage[absolute,overlay]{textpos}\textposquirk
\usepackage{tikz}
\usepackage{scalefnt}
\usepackage{listings}
\usepackage{fancyvrb}
\usepackage{hyperref}
   


% presentation title/author/etc.
\title{Virtualization with KVM(and Libvirt)}
%\subtitle{(subtitle)}
\institute{Red Hat}
\author{Kashyap Chamarthy}
\date{26 May 2012}


\fancySectionOpens
\fancyPartOpens
  
\begin{document}

\begin{rhbg}
\begin{frame}
    \titlepage

	% abstract, if you want.
%	\begin{abstract}
%	This is the abstract. Lorem ipsum dolor sit amet, consectetuer
%	adipiscing elit. Cras lacus. Praesent facilisis. Nam eget augue.
%	Nullam interdum sollicitudin est. Proin egestas venenatis ipsum.
%	Pellentesque vulputate laoreet elit. Nunc non sem. Nunc facilisis.
%	\end{abstract}
\end{frame}
\end{rhbg}


% agenda
\mode<presentation> {
    \begin{frame}
    \frametitle{Agenda}
    \tableofcontents
    \end{frame}
}

%section 1
\section{KVM Virtualization overview and architecture}


%kvm overview
\begin{frame}[fragile]
    \frametitle{KVM overview}
    \begin{itemize}
	\item Hardware-assisted 
        \begin{itemize}
        \item Uses CPU virtualization extensions(Intel-\textbf{vmx} /AMD-\textbf{svm})
        \item Enables a 'guest operating mode'
        \end{itemize}
    \item A loadable kernel module
    \begin{center}
    \scalefont{0.6}
%    \begin{exampleblock}{modprobing kvm modules}
    \begin{verbatim}
root@~$ modprobe -l | grep kvm
kernel/arch/x86/kvm/kvm.ko
kernel/arch/x86/kvm/kvm-intel.ko
kernel/arch/x86/kvm/kvm-amd.ko
root@~$ 
    \end{verbatim}
%    \end{exampleblock}
    \end{center}
 	\item Turns Kernel into hypervisor  
        \begin{itemize} %[<+->]
        \item \textbf{ /dev/kvm } character device is used by user-space to create guests
        \item Re-uses Linux kernel features --- CPU scheduling, Memory mgmt, Power mgmt,
        Drivers\ldots
        \item Design \& development driven by hardware specs
       \end{itemize} 
    \end{itemize}
\end{frame}


%kvm arch
\begin{frame}[fragile]
\frametitle{KVM Architecture}
    \scalefont{0.7}
%    \begin{verbatim}
%.-----------------------------------------------.
%|   .---.   .---.  .---------.   .---------.    |
%|   |AP1|   |AP2|  | Virtual |   | Virtual |    |
%|   '---'   '---'  | Machine |   | Machine |    |
%|   .---.   .---.  |   1     |   |    2    |    |
%|   |AP3|   |AP4|  '---------'   '---------'    |
%|   '---'   '---'  (VMs are also like           |
%|        APPs       regular linux processes)    |
%|                        .-------------.        |
%|                        |  QEMU(I/O)  |        |
%|      .----------------------------------.     |
%|      |                .-------------.   |     |
%|      |                |KVM(/dev/kvm)|   |     |
%|      |  (Kernel)      '-------------'   |     |
%|      '----------------------------------'     |
%'-----------------------------------------------'
%       .-----------------------------------.
%       |    x86 Hardware(VT-x/AMD-V)       |
%       '-----------------------------------'
%    \end{verbatim}
    \begin{verbatim}
.-------------------------------------------------------------.
|                                                             |
|                         [VMs are like regular               |
|                         linux processes]                    |
|                                                             |
|                        .--------------. .--------------.    |
|  .---------..--------. |   Virtual    | |   Virtual    |    |
|  |         ||        | |   Machine    | |   Machine    |    |
|  |  APP1   ||  APP2  | |      1       | |      2       |    |
|  |         ||        | |              | |              |    |
|  |         ||        | '--------------' .--------------.    |
|  |         ||        | | QEMU(I/O)    | | QEMU(I/O)    |    |
|  '---------''--------' |              | |              |    |
|                        |              | |              |    |
|                        '--------------' '--------------'    |
|                                |              |             |
|  .-----------------------------v--------------v---------.   |
|  |                         .--------------------.       |   |
|  |                         |    KVM(/dev/kvm)   |       |   |
|  | (Linux Kernel)          '--------------------'       |   |
|  '------------------------------------------------------'   |
'-------------------------------------------------------------'
      .--------------------------------------------------.
      |              x86 Hardware(VT-x/AMD-V)            |
      '--------------------------------------------------'
    \end{verbatim}
\end{frame}


 %section 2 
\section{libvirt, libguestfs and virt-tools} 

\begin{frame}[fragile]
\frametitle{Libvirt and Virt tools overview}
    \scalefont{0.8}
    \begin{verbatim}
.-------------.       .----------------------------.
| libguestfs  |       | virt-install; virsh; Boxes |.---------.
.-------------------. '-------------------------^--'|         |
| guestfish; virt-* |         |          |       \  |         |
'-------------------'         |          |        \ |         |
             |                v          v         \| Libvirt |
             |            .-------.   .-------.     |         |
             '----------> |  VM1  |   |  VM2  |     /         |
                          |       |   |       |    /|         |
                        .-'-------'---'-------'-. / |         |
                        |        QEMU(I/O)      |/  |         |
                        |                       v   '---------'
      .-----------------'-----------------------'
      |             Hypervisor - KVM(/dev/kvm)  |
      | Kernel                                  |
      '-----------------------------------------'
    \end{verbatim}
\end{frame}

%libvirt
\begin{frame}
\frametitle{Libvirt}
\begin{itemize}
    \item A library, daemon, tool to manage virtual\{machines, infrastructure\}:
    \begin{itemize} %[<+->
    \item General VM life-cycle management 
    \item XML format to store/define VM configuration
    \item Management shell interface -- \textbf{\alert{virsh}}
    \item CPU Pinning
    \item Manage: Devices, Networks, Snapshots, Storage pools\ldots
    \item Security: POSIX user/groups; sVirt(SELinux based)
    \item \alert{Plenty others\ldots}
    \end{itemize}
\end{itemize}
\end{frame}

%libguestfs
\begin{frame}
\frametitle{Libguestfs and virt-tools}
\begin{itemize}
    \item A library/API and a set of tools 
    \begin{itemize} %[<+->
    \item Inspect, edit, resize, rescue VM disk images
    \item \textbf{GUEST} \textbf{F}ilesystem \textbf{I}nteractive \textbf{SH}ell -- \textbf{\alert{guestfish}}
    \item Scriptable using \texttt{guestfish} or lang. bindings\{python, perl, ruby\ldots \}
    \item \textbf{libguestfs} based virt tools --- \texttt{virt-inspector, virt-cat, virt-edit,
    virt-rescue, virt-*}
    \end{itemize}
\end{itemize}
\end{frame}


%section 3 
\section{Demonstration}

%section 4 
\section{In the works/Upcoming}
\begin{frame}
    \frametitle{What's next}
\begin{itemize}
    \item Quite a few interesting things. A random list:
    \begin{itemize}%[<+->]
        \item Live snapshots/ Live migration
        \item Qcow2 (v3)
        \item Security: Sandboxing
        \item Storage : Virtio-scsi
        \item PCI Device assignment enhancements
        \item Nested Virt
        \item \alert{Plenty others\ldots}
    \end{itemize}
\end{itemize}
\end{frame}



%References
\begin{frame}
    \frametitle{References}

    \begin{thebibliography}{MMMMMMMMM}
        \bibitem{1} Fedora virt \\
        \url{https://fedoraproject.org/wiki/Getting_started_with_virtualization}
        \bibitem{2} KVM website\\
        \url{http://www.linux-kvm.org/}
        \bibitem{3} libvirt website \\
        \url{http://www.libvirt.org/}
        \bibitem{4} libguestfs \\
        \url{http://libguestfs.org/}
        \bibitem{5} virt-tools based on libvirt and libguestfs \\
        \url{http://virt-tools.org/}
        \bibitem{6} To reach Kashyap \\
        \texttt{kashyapc@fedoraproject.org}
        \url{https://kashyapc.wordpress.com}
    \end{thebibliography}
\end{frame}
\mode<presentation> {
    \Rhbg{\frame{\theend}}
}

\end{document}
